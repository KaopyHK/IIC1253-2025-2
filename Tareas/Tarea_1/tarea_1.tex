% Plantilla para documentos LaTeX para enunciados
% Por Pedro Pablo Aste Kompen - ppaste@uc.cl
% Licencia Creative Commons BY-NC-SA 3.0
% http://creativecommons.org/licenses/by-nc-sa/3.0/

\documentclass[12pt]{article}

% Margen de 1 pulgada por lado
\usepackage{fullpage}
% Incluye gráficas
\usepackage{graphicx}
% Packages para matemáticas, por la American Mathematical Society
\usepackage{amssymb}
\usepackage{amsmath}
% Desactivar hyphenation
\usepackage[none]{hyphenat}
% Saltar entre párrafos - sin sangrías
\usepackage{parskip}
% Español y UTF-8
\usepackage[spanish]{babel}
\usepackage[utf8]{inputenc}
% Links en el documento
\usepackage{hyperref}
\usepackage{fancyhdr}
\setlength{\headheight}{15.2pt}
\setlength{\headsep}{5pt}
\pagestyle{fancy}
% Código en el documento
\usepackage{listings}

\newcommand{\N}{\mathbb{N}}
\newcommand{\Exp}[1]{\mathcal{E}_{#1}}
\newcommand{\List}[1]{\mathcal{L}_{#1}}
\newcommand{\EN}{\Exp{\N}}
\newcommand{\LN}{\List{\N}}

\newcommand{\comment}[1]{}
\newcommand{\lb}{\\~\\}
\newcommand{\eop}{_{\square}}
\newcommand{\hsig}{\hat{\sigma}}
\newcommand{\ra}{\rightarrow}
\newcommand{\lra}{\leftrightarrow}

% Cambiar por nombre completo + número de alumno
\newcommand{\alumno}{Benjamín Andrés Gómez Maturana - 23628839}
\rhead{Tarea 1 --- \alumno}

\begin{document}
\thispagestyle{empty}
% Membrete
% PUC-ING-DCC-IIC1103
\begin{minipage}{2.3cm}
\includegraphics[width=2cm]{img/logo.pdf}
\vspace{0.5cm} % Altura de la corona del logo, así el texto queda alineado verticalmente con el círculo del logo.
\end{minipage}
\begin{minipage}{\linewidth}
\textsc{\raggedright \footnotesize
Pontifícia Universidad Católica de Chile \\
Departamento de Ciencias de la Computación \\
IIC1253 --- Matemáticas Discretas \\}
\end{minipage}


% Titulo
\begin{center}
\vspace{0.5cm}
{\huge\bf Tarea N}\\
\vspace{0.2cm}
\today\\
\vspace{0.2cm}
\footnotesize{2º semestre 2025 - Profesores M. Arenas - A. Kozachinskiy - M. Romero}\\
\vspace{0.2cm}
\footnotesize{\alumno}
\rule{\textwidth}{0.05mm}
\end{center}



\section*{Respuestas}
% Estas numeracion es solo de ejemplo

%\subsection*{Pregunta 1}
\subsubsection*{Pregunta \text{1 --- (a)}}


% ---------------------- Inicio pregunta 1 ----------------------


% ---- Pregunta 1 - (a)

Encuentre una fórmula en CNF que sea equivalente a:
\[ (p \rightarrow (q \rightarrow (r \to s))). \]
Debe explicar claramente su desarrollo para obtener la fórmula.\\
\rule{\linewidth}{0.2pt} % Línea del ancho del texto y grosor 0.4pt

% --> Respuesta:
Mediante demostración semántica o tablas de verdad sabemos que:
\[ (A \rightarrow B \equiv \neg A \lor B )\]
Lo cual indica que estas dos expresiones son lógicamente equivalentes, por lo tanto:
\[ (p \rightarrow (q \rightarrow (\neg r \lor s)))\] % Paso 1
\[ (p \rightarrow (\neg q \lor (\neg r \lor s)))\] % Paso 2
\[ (\neg p \lor (\neg q \lor (\neg r \lor s)))\] % Paso 3
Por lo tanto, y debido a la propiedad asociativa del conectivo V (OR) nos queda:
\[ \neg p \lor \neg q \lor \neg r \lor s\]
Por lo que se llega a una CNF, ya que la expresión anterior es una disyuntiva literal, es decir,
una sola cláusula disyuntiva

% ---- Fin Pregunta 1 - (a)

% ---- Pregunta 1 - (b)

\subsubsection*{Pregunta \text{1 --- (b)}}

Definimos el conectivo binario XOR según la siguiente tabla de verdad:
\[
\begin{array}{c|c|c}
p & q & \text{XOR} \\ \hline 
0 & 0 & 0 \\ 
0 & 1 & 1 \\ 
1 & 0 & 1 \\ 
1 & 1 & 0 
\end{array}
\]
Demuestre que el conjunto $\{\text{XOR}, \rightarrow\}$ es funcionalmente completo.
\\ (Hint: Recuerde que el conjunto $\{\neg, \rightarrow \}$ es funcionalmente completo.)
\\ \rule{\linewidth}{0.4pt} % Línea del ancho del texto y grosor 0.4pt

% --> Respuesta:


\rule{\linewidth}{0.4pt}
% ---- Fin Pregunta 1 - (b)

% ---- Pregunta 1 - (c)

\subsubsection*{Pregunta \text{1 --- (c)}}


Demuestre que el conjunto $\{\text{XOR}\}$ no es funcionalmente completo.
\\ (Hint: Demuestre que la fórmula $\{\neg p \}$ no se puede expresar utilizando sólo XOR)
\\ \rule{\linewidth}{0.4pt}

% --> Respuesta:

% ---- Fin Pregunta 1 - (c)


% ---------------------- FIN Pregunta 1 ----------------------


% ---------------------- Inicio pregunta 2 ----------------------


\newpage
%\subsection*{Pregunta 2}

\subsubsection*{Pregunta \text{2 --- (a)}}
\rule{\linewidth}{0.4pt} % Línea del ancho del texto y grosor 0.4pt


\subsubsection*{Pregunta \text{2 --- (b)}}
\rule{\linewidth}{0.4pt} % Línea del ancho del texto y grosor 0.4pt


% ---------------------- FIN Pregunta 2 ----------------------

Si necesita adjuntar código en su documento puede hacerlo de la siguiente manera:

\begin{lstlisting}[language=Python]
print("Hello, Discretas!")
\end{lstlisting}

Existen otras maneras de adjuntar código, como por ejemplo referenciar un archivo de código y especificarlo como parámetro. También se puede cambiar el estilo en el que se muestra el codigo en el documento compilado. Más información sobre adjuntar código en \href{https://www.overleaf.com/learn/latex/Code_listing}{https://www.overleaf.com/learn/latex/Code\_listing}







% Fin del documento
\end{document}
