% Plantilla para documentos LaTeX para enunciados
% Por Pedro Pablo Aste Kompen - ppaste@uc.cl
% Licencia Creative Commons BY-NC-SA 3.0
% http://creativecommons.org/licenses/by-nc-sa/3.0/

\documentclass[12pt]{article}

% Margen de 1 pulgada por lado
\usepackage{fullpage}
% Incluye gráficas
\usepackage{graphicx}
% Packages para matemáticas, por la American Mathematical Society
\usepackage{amssymb}
\usepackage{amsmath}
% Desactivar hyphenation
\usepackage[none]{hyphenat}
% Saltar entre párrafos - sin sangrías
\usepackage{parskip}
% Español y UTF-8
\usepackage[spanish]{babel}
\usepackage[utf8]{inputenc}
% Links en el documento
\usepackage{hyperref}
\usepackage{fancyhdr}
\setlength{\headheight}{15.2pt}
\setlength{\headsep}{5pt}
\pagestyle{fancy}
% Código en el documento
\usepackage{listings}

\newcommand{\N}{\mathbb{N}}
\newcommand{\Exp}[1]{\mathcal{E}_{#1}}
\newcommand{\List}[1]{\mathcal{L}_{#1}}
\newcommand{\EN}{\Exp{\N}}
\newcommand{\LN}{\List{\N}}

\newcommand{\comment}[1]{}
\newcommand{\lb}{\\~\\}
\newcommand{\eop}{_{\square}}
\newcommand{\hsig}{\hat{\sigma}}
\newcommand{\ra}{\rightarrow}
\newcommand{\lra}{\leftrightarrow}

% Cambiar por nombre completo + número de alumno
\newcommand{\alumno}{Benjamín Andrés Gómez Maturana - 23628839}
\rhead{Tarea 1 --- \alumno}

\begin{document}
\thispagestyle{empty}
% Membrete
% PUC-ING-DCC-IIC1103
\begin{minipage}{2.3cm}
\includegraphics[width=2cm]{img/logo.pdf}
\vspace{0.5cm} % Altura de la corona del logo, así el texto queda alineado verticalmente con el círculo del logo.
\end{minipage}
\begin{minipage}{\linewidth}
\textsc{\raggedright \footnotesize
Pontifícia Universidad Católica de Chile \\
Departamento de Ciencias de la Computación \\
IIC1253 --- Matemáticas Discretas \\}
\end{minipage}


% Titulo
\begin{center}
\vspace{0.5cm}
{\huge\bf Tarea N}\\
\vspace{0.2cm}
\today\\
\vspace{0.2cm}
\footnotesize{2º semestre 2025 - Profesores M. Arenas - A. Kozachinskiy - M. Romero}\\
\vspace{0.2cm}
\footnotesize{\alumno}
\rule{\textwidth}{0.05mm}
\end{center}



\section*{Respuestas}
% Estas numeracion es solo de ejemplo

%\subsection*{Pregunta 1}
\subsubsection*{Pregunta \text{1 --- (a)}}


% ---------------------- Inicio pregunta 1 ----------------------


% ---- Pregunta 1 - (a)

Mediante demostración semántica o tablas de verdad sabemos que:
\[ (A \rightarrow B \equiv \neg A \lor B )\]
Lo cual indica qe estas dos expresiones son lógicamente equivalentes, por lo tanto:
\[ (p \rightarrow (q \rightarrow (\neg r \lor s)))\] % Paso 1
\[ (p \rightarrow (\neg q \lor (\neg r \lor s)))\] % Paso 2
\[ (\neg p \lor (\neg q \lor (\neg r \lor s)))\] % Paso 3
Por lo tanto, y debido a la propiedad asociativa del conectivo V (OR) nos queda:
\[ \neg p \lor \neg q \lor \neg r \lor s\]
Por lo que se llega a una CNF, ya que la expresión anterior es una disyuntiva literal, es decir,
una sola cláusula disyuntiva

\rule{\linewidth}{0.4pt}

% ---- Fin Pregunta 1 - (a)

% ---- Pregunta 1 - (b)

\subsubsection*{Pregunta \text{1 --- (b)}}

Para demostrar que un conjunto de X funciones booleanas es funcionalmente completo, debemos demostrar
que se puede, utilizando sólo estos conectivos, expresar cualquier otra función booleana.
\\Para este ejercicio en particular nos es de utilidad saber que el conjunto $\{\neg, \rightarrow \}$ ya es funcionalmente completo.
\\ Tabla verdad XOR: $(A \rightarrow B) = (\neg A \lor B)$
\[XOR = (\neg p \land q) \lor (p \land \neg q)\]
\[(\neg p \land q) \lor (p \land \neg q) = (\neg p \rightarrow \neg q) \rightarrow \neg(p \rightarrow q)\]
\[
\begin{array}{c|c|c|c|c}
p & q & XOR & (\neg p \land q) \lor (p \land \neg q) & (\neg p \rightarrow \neg q) \rightarrow \neg(p \rightarrow q)\\ \hline
0 & 0 & 0 & 0 & 0\\
1 & 0 & 1 & 1 & 1\\
1 & 1 & 0 & 0 & 0\\
0 & 1 & 1 & 1 & 0\\
\end{array}
\]
Mediante lo anterior demostramos que con la funcion XOR podemos expresar los conectivos $(\neg, \lor)$,
a su vez, con estos conectivos expresamos $\rightarrow$, por lo que queda demostrado que con $\{XOR, \rightarrow\}$
se puede expresar $\{\neg, \rightarrow\}$ que ya es un conjunto funcionalmente completo.

% ---- Fin Pregunta 1 - (b)

% ---- Pregunta 1 - (c)

\subsubsection*{Pregunta \text{1 --- (c)}}


\rule{\linewidth}{0.4pt}

% ---- Fin Pregunta 1 - (c)
% ---------------------- FIN Pregunta 1 ----------------------


% ---------------------- Inicio pregunta 2 ----------------------

% \newpage
%\subsection*{Pregunta 2}

% ---- Pregunta 2 - (a)

\subsubsection*{Pregunta \text{2 --- (a)}}

El problema del \textit{cuadrado latino} de 4 x 4 se define como sigue. Tenemos un 
tablero de 4 x 4 casillas. Algunas de las casillas están \textit{ocupadas}, vale decir, tienen un
número entre $\{1, 2, 3, 4\}$. El resto de las casillas están \textit{libres}. El objetivo es verificar
si existe una \textit{solución}, esto es, una forma de asignarle números entre $\{1, 2, 3, 4\}$ a las
casillas libres, tal que en cada una de las 4 filas y en cada una de las 4 columnas, los números 
que aparecen sean distintos. Por ejemplo, una posible instancia al problema puede ser el siguiente tablero:

\[
\begin{array}{|c|c|c|c|} \hline
 &  & \textbf{1} &  \\ \hline
 & \textbf{3} &  &  \\ \hline
 &  &  & \textbf{4} \\ \hline
\textbf{2} &  & \textbf{3} &  \\ \hline
\end{array}
\]
Una posible solución es la siguiente:
\[
\begin{array}{|c|c|c|c|} \hline
4 & 2 & \textbf{1} & 3 \\ \hline
1 & \textbf{3} & 4 & 2 \\ \hline
3 & 1 & 2 & \textbf{4} \\ \hline
\textbf{2} & 4 & \textbf{3} & 1 \\ \hline
\end{array}
\]
Por otra parte, el siguiente tablero \textbf{no} tiene solución (verifíquelo):
\[
\begin{array}{|c|c|c|c|} \hline
 &  & \textbf{1} &  \\ \hline
 & \textbf{3} &  & \textbf{1} \\ \hline
 &  &  & \textbf{4} \\ \hline
\textbf{2} &  & \textbf{3} &  \\ \hline
\end{array}
\]
Para describir el conjunto de casillas ocupadas usaremos triples de la siguiente forma: un triple
(\textit{i, j, k}), donde 1 $\leq$ \textit{i, j, k} $\leq$ 4, indica que las casillas en la fila \textit{i}
y columna \textit{j} está ocupada con el número \textit{k}. Por ejemplo, las casillas ocupadas del primer
ejemplo quedan descritas por:
\[\{(1, 3, 1), (2, 2, 3), (3, 1, 4), (4, 2, 2), (4, 3, 3)\},\]
mientras que en el segundo ejemplo quedan descritas por:
\[\{(1, 3, 1), (2, 2, 3), (2, 4, 1), (3, 4, 4), (4, 1, 2), (4, 3, 3)\}.\]
Dado un tablero con casillas ocupadas $\{(i_1, j_1, k_1),..., (i_m, j_m, k_m)\}$, escriba una fórmula
$\varphi$ en la lógica proposicional tal que:\\
\center{el tablero tiene solución si y sólo si $\phi$ es satisfacible.}\\
Para esto SOLO debe utilizar variables proposicionales $x_{i,j,k,}$ donde 1 $\leq$ \textit{i, j, k} $\leq$ 4, que
indican que la casilla de la fila \textit{i} y la columna \textit{j} recibe el número \textit{k}.

\rule{\linewidth}{0.4pt} % Línea del ancho del texto y grosor 0.4pt

% --> Respuesta:
RESPUESTA\\AAAA\\AAAA\\AAAA
% ---- Fin pregunta 2 - (a)

\rule{\linewidth}{0.4pt} % Línea del ancho del texto y grosor 0.4pt

% ---- Pregunta 2 - (b)
\subsubsection*{Pregunta \text{2 --- (b)}}

Utilizando su fórmula proposicional de la parte anterior y el solver \textbf{Z3}, encuentre una
solución para el siguiente tablero:
\[
\begin{array}{|c|c|c|c|} \hline
\textbf{2} & & &  \\ \hline
 & & \textbf{1} & \\ \hline
 & \textbf{4} & & \textbf{3} \\ \hline
\textbf{3} &  &  & \\ \hline
\end{array}
\]
Debe pegar su código de Python junto con la solución obtenida en su documento LaTeX. Recuerde también
subir su archivo .py como fue indicado en las instrucciones.

\rule{\linewidth}{0.4pt} % Línea del ancho del texto y grosor 0.4pt

% --> Respuesta:


% ---------------------- FIN Pregunta 2 ----------------------

Si necesita adjuntar código en su documento puede hacerlo de la siguiente manera:

\begin{lstlisting}[language=Python]
print("Hello, Discretas!")
\end{lstlisting}

Existen otras maneras de adjuntar código, como por ejemplo referenciar un archivo de código y especificarlo como parámetro. También se puede cambiar el estilo en el que se muestra el codigo en el documento compilado. Más información sobre adjuntar código en \href{https://www.overleaf.com/learn/latex/Code_listing}{https://www.overleaf.com/learn/latex/Code\_listing}







% Fin del documento
\end{document}
